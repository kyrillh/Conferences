\documentclass[12pt]{eccomas-2024}

%\usepackage{graphicx}
%\usepackage{amsmath}
%\usepackage{amsfonts}
%\usepackage{amssymb}

\title{Parallel implementation of two-level nonlinear Schwarz domain decomposition methods\\}

\author{K. Ho*$^{1}$, A. Klawonn$^{1,2}$, M. Lanser$^{1,2}$}

%\author{First A. Author$^{1}$, Second B. Author$^{2}$ and Third C. Author$^{3}$}

\address{$^{1}$ Center for Data and Simulation Science (CDS)\\ University of Cologne
\and
$^{2}$ Department of Mathematics and Computer Science\\University of Cologne\\
\{kyrill.ho, axel.klawonn, mlanser\}@uni-koeln.de, www.numerik.uni-koeln.de
}

%\address{$^{1}$ Affiliation, Postal Address, E-mail address and URL
%\and
%$^{2}$ Affiliation, Postal Address, E-mail address and URL
%\and
%$^{3}$ Affiliation, Postal Address, E-mail address and URL}

\begin{document}

\noindent {\bf Keywords}: {\it  Nonlinear Domain Decomposition Methods, Nonlinear Schwarz, Trilinos, Parallel Computing}
\vskip0.5cm

Domain decomposition methods (DDMs) are a class of robust and easily parallelizable iterative solvers. They are best known as preconditioners for solving linear discretized partial differential equations, where they achieve parallelization through a divide-and-conquer approach by decomposing the computational domain into subdomains. When solving nonlinear partial differential equations, linear DDMs are employed as a preconditioner for the Krylov subspace method used to solve the tangential system in each iteration of a Newton-type method.

Nonlinear DDMs are alternatives to classical Newton-Krylov-DDMs that apply a domain decomposition before linearization and solve nonlinear problems locally and globally. In this sense, they can be interpreted as nonlinear preconditioners to Newton's method and have been shown to improve its nonlinear convergence performance. In addition, the two-level nonlinear DDM variants hold the potential for excellent scalability. These favorable properties come at the cost of a high implementation complexity, which has resulted in most research being based on sequential implementations or implementations that scale to less than a hundred subdomains. We are working on a highly scalable implementation of one- and two-level nonlinear Schwarz methods based on FROSch, a sub-package of the Trilinos project which implements linear one and multi-level Schwarz DDMs.

In this talk, we present results from our two-level nonlinear Schwarz implementation, showcasing its nonlinear convergence and parallel scalability when applied to a nonlinear elasticity problem and the lid-driven cavity problem for high Reynolds numbers. We discuss key insights we have acquired during the implementation and performance investigation phases and consider future areas of deployment of our implementation.

% \begin{thebibliography}{99}
% \bibitem{Heinlein1}
% A. Heinlein and M. Lanser. Additive and Hybrid Nonlinear Two-Level Schwarz Methods and Energy Minimizing Coarse Spaces for Unstructured Grids. SIAM Journal on Scientific Computing 42 (4): A2461-A2488, 2020.
% \bibitem{Heinlein2}
% A. Heinlein, A. Klawonn and M. Lanser. Adaptive Nonlinear Domain Decomposition Methods with an Application to the $p$-Laplacian. SIAM Journal on Scientific Computing 45(3): S152-S172, 2023. 
% \bibitem{Heinlein3} A. Heinlein, C. Hochmuth, A. Klawonn. Monolithic Overlapping Schwarz Domain Decomposition Methods with GDSW Coarse Spaces for Incompressible Fluid Flow Problems. SIAM Journal on Scientific Computing 41(4): C291-C316, 2019. 
% \bibitem{Heinlein4} A. Heinlein, A. Klawonn, S. Rajamanickam, and O. Rheinbach. \textit{FROSch: A Fast And Robust Overlapping Schwarz Domain Decomposition Preconditioner Based on Xpetra in Trilinos}.  Springer Nature Switzerland AG, 2020
% \end{thebibliography}

\end{document}


